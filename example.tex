\documentclass{beamer}
\usepackage[utf8]{inputenc}
\usepackage{graphicx}
\usetheme{UoE}
\usepackage{amsmath}

% This file contains an example of how to use the
% University of Edinburgh beamer template, available at
% https://github.com/mthulin/UoE-beamer

%%%%%%%%%%%%%%%%%%%%%%%%%%%%%%%%%%%%%%%%%%%%%%%%%%%%%%%%%%%%%%%%%%

\title[]{{Introduction to the Doctrine of Fluxions}}
\author[]{\textbf{Thomas Bayes}\\School of Mathematics and Maxwell Institute for Mathematical Sciences\\University of Edinburgh}
\date{ }

\begin{document}


\frame{\titlepage}

%%%%%%%%%%%%%%%%%%%%%%%%%%%%%%%%%%%%%%%%%%%%%%%%%%%%%%%%%%%%%%%%%%

\section{Main result}
\frame{\frametitle{Main result}
Under the Peano axioms,
\[ 1+1=2 \]
}

\section{Proof sketch}
\frame{\frametitle{Sketch of proof}
As always, the proof involves a second order asymptotic expansion and inverting a Laplace transform.\\[3mm]
An Essay towards solving a Problem in the Doctrine of Chances. An Essay towards solving a Problem in the Doctrine of Chances. An Essay towards solving a Problem in the Doctrine of Chances.
$$\int_0^\infty \frac{1}{x^2}dx$$
An Essay towards solving a Problem in the Doctrine of Chances...?
}

\section{Detailed proof}
\frame{\frametitle{Detailed proof}
More details. The audience loves the details. Always include detailed details about your proof.\\[3mm]
More details. The audience loves the details. Always include detailed details about your proof.\\[3mm]
More details. The audience loves the details. Always include detailed details about your proof.\\[3mm]
More details. The audience loves the details. Always include detailed details about your proof.
}

\section{Tedious proofs of 5 humdrum lemmata}
\frame{\frametitle{Further details}
Also, make sure to include proofs of technical lemmas. Omitting them will make it clear to the audience that you are an imposter.
}


\end{document}
